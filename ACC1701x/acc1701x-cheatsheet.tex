\documentclass{article}
\usepackage[a4paper, margin=3mm, landscape]{geometry}
\usepackage{multicol}
\usepackage{xcolor}
\usepackage{enumitem}
\usepackage{amsmath}
\usepackage{amsfonts}
\usepackage{listings}
\usepackage{soul}
\usepackage{graphicx}

\pdfinfo{
    /Title (ACC1701x.pdf)
    /Creator (TeX)
    /Producer (pdfTeX 1.40.0)
    /Author (Vincent Pang)
    /Subject (ACC1701x)
    /Keywords (ACC1701x, nus, cheatsheet, pdf)
}

\graphicspath{ {./img/} }

\pagestyle{empty}
\setcounter{secnumdepth}{0}
\setlength{\columnseprule}{0.25pt}

% Redefine section commands to use less space
\makeatletter
\renewcommand{\section}{\@startsection{section}{1}{0mm}%
    {-1ex plus -.5ex minus -.2ex}%
    {0.5ex plus .2ex}%x
{\normalfont\large\bfseries}}
\renewcommand{\subsection}{\@startsection{subsection}{2}{0mm}%
    {-1explus -.5ex minus -.2ex}%
    {0.5ex plus .2ex}%
{\normalfont\normalsize\bfseries}}
\renewcommand{\subsubsection}{\@startsection{subsubsection}{3}{0mm}%
    {-1ex plus -.5ex minus -.2ex}%
    {1ex plus .2ex}%
{\normalfont\small\bfseries}}%
\makeatother

% Adjust spacing for all itemize/enumerate
\setlength{\leftmargini}{0.5cm}
\setlength{\leftmarginii}{0.5cm}
\setlist[itemize,1]{leftmargin=2mm,labelindent=1mm,labelsep=1mm}
\setlist[itemize,2]{leftmargin=2mm,labelindent=1mm,labelsep=1mm}

% Font
\renewcommand{\familydefault}{\sfdefault}

% Define colors for math formulas
\definecolor{myblue}{cmyk}{1,.72,0,.38}
\everymath\expandafter{\the\everymath \color{myblue}}

% Custom command for keywords
\definecolor{highlight}{RGB}{251,243,218}
\newcommand{\keyword}[2][]{\sethlcolor{highlight}\hl{\textbf{#2}} #1 - }
\newcommand{\ilkeyword}[1]{\sethlcolor{highlight}\hl{\textbf{#1}}}

% Define colors and style for code
\definecolor{codegreen}{rgb}{0,0.6,0}
\definecolor{codegray}{rgb}{0.5,0.5,0.5}
\definecolor{codered}{HTML}{CC241D}
\definecolor{backcolor}{rgb}{0.95,0.95,0.95}
\lstdefinestyle{codestyle}{
    backgroundcolor = \color{backcolor},
    commentstyle = \color{codegray},
    keywordstyle = \color{codered},
    stringstyle = \color{codegreen},
    basicstyle = \ttfamily,
    breakatwhitespace = false,
    showstringspaces = false,
    breaklines = true,
    showtabs = false,
    tabsize = 2
}
\lstset{style = codestyle}

% -----------------------------------------------------------------------
\begin{document}
\begin{multicols*}{3}
\footnotesize

% Title box
\begin{center}
    \fbox{
        \parbox{0.8\linewidth}{
            \centering \textcolor{black}{
                {\Large\textbf{ACC1701x}} \\
                \normalsize{AY22/23 Sem 2}} \\
                {\footnotesize \textcolor{gray}{github.com/securespider}}
        }
    }
\end{center}
%-----------------------------------------------------------------------------------------------------------------------
\section{01. Financial Statements}
\keyword{Consolidated Financial Statements}{Financial information about the group of companies including parent and subsidiaries}
\\How does a parent company control subsidiary?
\begin{itemize}
	\item Own a controlling interest of subsidiary's share
	\item Remaining shares are \keyword{Non controlling interest}{Separated, under book value}
\end{itemize}
\keyword{Associate Companies}{Interest in shares outside group that wields significant influence}
\begin{itemize}
	\item Includes one-line partial consolidation of associate company in book value
	\item Recorded using equity method
\end{itemize}

\subsection{Consolidated Balance Sheet}
Purpose
\begin{itemize}
	\item Report net worth of group \textbf{at specific date}
\end{itemize}
\subsubsection{Fundamental Accounting Equation} 
\begin{description}
	\item \textit{Assets = Liabilities + Equity}
	\item[Assets]{Resource controlled by company (eg. cash, accounts receivable, inventory, land, equipment, buildings)}
	\item[Liabilities]{Amount owed to others leading to outflow of resources (eg. accounts payable, expenses)}
	\item[Equity]{Owner's claim on residual interest after deducting liabilities}
\end{description}
\subsubsection{Problems}
\begin{enumerate}
	\item Not market value
	\begin{itemize}
		\item Does not tell what the equity is worth in market
		\item Market value = market share price * outstanding shares
	\end{itemize}
	\item Mixed measurement model - Mathematically dubious calculation 
	\begin{itemize}
		\item Property measured using \keyword{cost less depreciation}{amount originally paid less depreciation}, OR current market value
	\end{itemize}
\end{enumerate}
\subsubsection{Motivation}
\begin{enumerate}
	\item Comparing
	\begin{itemize}
		\item Compare companies using \keyword{Market-to-book ratio}{$\dfrac{Market~value}{Total~equity}$}
		\item Why are they not the same?
		\begin{itemize}
			\item Market takes into account future prospects not captured by book value
		\end{itemize}
	\end{itemize}
	
	\item Details
	\begin{itemize}
		\item Provides useful information
	\end{itemize}
\end{enumerate}

\subsection{Statement of Changes in Equity}
Lists impact of events on changes in equity
\subsubsection{Factors}
\begin{enumerate}
	\item Contributions from shareholders
	\item Distributions to shareholders
	\item Business income/expenses
	\item \keyword{Capital maintenance adjustments}{Remeasurement of asset/liability value}
\end{enumerate}

\subsection{Statement of Comprehensive Income}
\begin{itemize}
	\item \keyword{Comprehensive Income}{Reflects changes of equity from non-owner sources and traditional income}
	\item Show all operating and financial events that affect non-owners' interest in business
	\item Includes unrealised gains and losses

\end{itemize}

\subsubsection{Income Statement}
Includes information about business income and expenses for \textbf{the year}

\subsection{Statement of Cash Flows}
\keyword{Accrual Basis}{Record values when exchange of goods and services \textbf{NOT cash flows}}
\subsubsection{Subsections}
\begin{itemize}
	\item Operating activities
	\item Investing activities
	\item \keyword{Financing activities}{Borrowing or issuing shares (eg. repayments, share buybacks)}
\end{itemize}

\subsection{Item Breakdown}
\begin{description}
	\item[Current assets/liabilities]{Likely to be converted to cash/settled within a year}
	\item[PPE]{Property, plant and equipment used in business}
	\item[Right-of-use assets]{Rented premises for business (Asset)}
	\item[Lease liabilities]{Outstanding rental payments wrt ROU assets}
	\item[Trade Receivables]{Outstanding dues from \textbf{credit} customers}
	\item[Cost of Goods Sold(COGS)]{Amount paid to suppliers for goods sold to customers}
	\item[Gross Profit]{= COGS - Sales}
	\item[Interest Income and expense]{Profit from investing and financing activities}
\end{description}
%-----------------------------------------------------------------------------------------------------------------------
\section{02. Ratios}
\begin{itemize}
	\item Comparing companies as investment opportunities
	\item Comparing previous years to measure progress
	\item Overcome difference in scale
\end{itemize}

\subsection{Profitability Ratios}
\begin{itemize}
	\item Measures performance of company over the year
\end{itemize}
\begin{description}
	\item[Profit Margin]{$\dfrac{Net~Profits}{Net~sales~(revenue)}$}
	\item[Return on total assets]{$\dfrac{Net~Profit}{Average~Total~Assets}$}
	\item[Return on ordinary shareholders' equity]{$\dfrac{Net~Profit}{Average~total~equity}$}
	\item[Earnings per share]{$\dfrac{Ordinary~shareholders~profits-Pref~Dividends}{Weighted~average~number~of~shares~during~year}$}
\end{description}
\subsubsection{Equity Categories}
\begin{itemize}
	\item \keyword{Share capital}{Amount collected when company originally issued shares}
	\item \keyword{Retained earnings}{Accumulated profits - Amount paid as dividents}
\end{itemize}

\subsection{Liquidity and efficiency ratios}
\begin{itemize}
	\item Examine company capacity to meet short term debt obligation with current assets
\end{itemize}
\begin{description}
	\item[Current ratio]{$\dfrac{Current~assets}{Current~liabilities}$}
	\item[Acid Test ratio]{$\dfrac{Cash+Short~term~fin~assets+Current~receivables}{Current~liabilities}$}
	\item[Accounts receivable turnover]{$\dfrac{Net~Sales}{Average~accounts~(Trade)~receivables}$}
	\begin{itemize}
		\item Receivables is \keyword{net}{Adjustment made for customers who may default}
	\end{itemize}
	\item[Inventory Turnover]{$\dfrac{Cost~of~goods~sold}{Average~inventory}$}
	\item[Accounts payable turnover]{$\dfrac{Cost~of~goods~sold}{Average~accounts~(trade)~payable}$}
	\item[Days' sales uncollected]{$\dfrac{Accounts~(Trade)~receivables,~net}{Net~sales}*365$}
	\item[Days' sales in inventory]{$\dfrac{Ending~Inventory}{Cost~of~goods~sold}*365$}
	\item[Days' purchases in accounts payable]{$\dfrac{Account~(trade)~payable}{Cost~of~goods~sold}*365$}
	\item[Total Asset Turnover]{$\dfrac{Net~sales}{Average~total~assets}$}
\end{description}

\subsubsection{Prepaid flows}
\begin{itemize}
	\item \keyword{Prepaid Expense}{Payments made for services not yet received (Assets)}
	\item \keyword{Unearned Revenue}{Advanced payments received from customers (Liabilities)}
\end{itemize}

\subsection{Solvency Ratios}
\begin{itemize}
	\item Identify the company's risk of going bankrupt
	\item Gauge company chances of staying afloat
\end{itemize}
\begin{description}
	\item[Debt Ratio]{$\dfrac{Total~liabilities}{Total~assets}$}
	\item[Equity Ratio]{$\dfrac{Total~equity}{Total~assets}$}
	\item[Times interest earned]{$\dfrac{Profit~before~interest~expense/tax}{Interes~expense}$}
	\item[Debt to equity ratio]{$\dfrac{Total~Liabilities}{Total~equity}$}
\end{description}

\subsection{Market prospects ratio}
\begin{itemize}
	\item Help compare share price to other investments
\end{itemize}

\begin{description}
	\item[Price-earnings ratio]{$\dfrac{Market~price~per~ordinary~share}{Earnings~per~share}$}
	\item[Divident yield]{$\dfrac{Annual~cash~dividents~per~share}{Market~price~per~share}$}
\end{description}

%-----------------------------------------------------------------------------------------------------------------------
\newpage
\section{03. Accounting Equation}
\keyword{Sole Proprietorship}{Business owned by a single party}
$\triangle Capital=Capital~contributed+Income-Expenses+Withdrawals$
\subsection{Concepts}
\begin{itemize}
	\item \keyword{Accrual accounting}{Revenue and expenses recorded when goods and services change hands}
	\item \keyword{Income}{Increase in equity net from any contributions}
	\item \keyword{Expense}{Decrease in equity net from withdrawals}
\end{itemize}
\begin{enumerate}
	\item{Income statement}{Records revenues, expenses and net profit}
	\item{Statements of changes in equity}
	\begin{itemize}
		\item{Records changes in a month}
		\item{Includes contributions from owners and withdrawals}
	\end{itemize}
	\item Balance sheet
	\begin{itemize}
		\item{Refers to a specific date so must specify date}
		\item{Breakdown assets and liabilities}
	\end{itemize}
	\item Statements of cash flow
	\begin{description}
		\item{Differentiate the types of cash flows}
		\item[Operating]{Day-to-day operations of company}
		\item[Investing]{Cash used to buy/receive from sales of long lived assets used in business}
		\item[Financing]{Borrowing/lending or cash flows from owner}
	\end{description}	
\end{enumerate}

%-----------------------------------------------------------------------------------------------------------------------
\section{04. Debit and Credit}
Represented via a T-account
\\\includegraphics[scale=0.5]{t-account}
\begin{itemize}
	\item Every transaction there will be equal amounts listed
	\item Debit: Asset increase (eg. Withdrawals, Expenses)
	\item Credit: Liability, equity increase (eg. Revenue)
\end{itemize}
\begin{description}
	\item Permanent vs temporary accounts
	\item[Temporary]{Track accounts only during current period}
	\item[Permanent]{Capital accounts that track equity long-term}
	\item[Trade debtors]{Customer that has not paid for goods and services}
	\item[Trade creditor]{Supplier who has sent your business goods/services but haven't paid}
\end{description}
\keyword{Journal entries}{Convenient format for recording transactions}
%-----------------------------------------------------------------------------------------------------------------------
\section{05. Adjusting and closing}
\subsection{Adjusting entries}
Record of transactions happening during period that were unrecorded (Why unrecorded)
\begin{enumerate}
	\item Frequent or continuous transactions - impractical to record 
	\item Earning revenue/incurring expenses do not happen during cash payments (Prepayments/ accrued revenues and expenses)
	\item \keyword{Depreciation}{Adjustment for age of a long-lived asset used in business}
	\begin{tabbing}
		\= \keyword{Accumulated Depreciation}{Contra asset storing negative adjustment}
		\\\= Goes up on credit side and down on debit side\\
		\= \keyword{Carrying value}{Remaining value of asset (cost - accumulated depreciation)}
	\end{tabbing}
\end{enumerate}

\subsection{Closing entries}
Clear the temporary accounts
\\Capital(post-closing) = Capital(pre-closing) + Revenues - Expenses - Withdrawals
\subsubsection{Steps}
\begin{enumerate}
	\item Close revenue accounts to income summary (Debit revenue, credit income summary)
	\item Close expense accounts to income summary (Credit expense, debit income summary)
	\item Close income summary to capital
	\begin{itemize}
		\item Debit $>$ Credit: Debit balance $\rightarrow$ Net loss (Credit income summary balance, debit capital)
		\item Credit $>$ Debit: Credit balance $\rightarrow$ Net profit (Debit income summary balance, credit capital)
	\end{itemize}
	\item Close withdrawals account directly to Capital (Credit withdrawals, debit capital)
\end{enumerate}
%-----------------------------------------------------------------------------------------------------------------------
\section{06. Inventory}
\subsection{Sales and cost of goods sold}
\subsubsection{Transactions}
\begin{enumerate}
	\item Earned revenue (Sales revenue and cash/accounts receievable increase)
	\item Goods sold (Inventory and COGS expense decrease)
\end{enumerate}
\subsubsection{Perpetual System}
\begin{itemize}
	\item Tracks on every sale 
	\item vs \keyword{Periodic}{Compare beginning + purchases and ending inventory}
\end{itemize}
\subsection{Shrinkage}
Inventory is less than beginning (Unaccounted damage, loss)
\subsubsection{Min(Cost, NRV)}
\keyword{Cost}{Cost to acquire and make inventory available for sale}
\begin{itemize}
	\item Purchase price, NET of discounts or allowance
	\item Shipping cost (\keyword{Freight-in}{borne by the buyer})
	\item Taxes on purchase transaction (as long as not recoverable)
\end{itemize}
\keyword{Net Realizable Value}{Value that inventory can be sold, net reasonable cost to sell}
\subsection{Cash flow assumption}
\begin{itemize}
	\item \keyword{Specific identification}{Unique items that have to be tracked and use actual original cost of item}
	\item \keyword{Interchangeable goods}{Indistinguishable goods, typically in bulk}
	\begin{enumerate}
		\item FIFO
		\item Weighted average cost - Cost of each good are the same at the time of sale
	\end{enumerate}
\end{itemize}


%-----------------------------------------------------------------------------------------------------------------------

\section{Midterm}
Debits
\begin{description}
	\item[Assets]{Cash, Account receivables. Supplies/inventory, \textbf{PREPAID} insurance/rent, equipment, land, buildings}
	\item[Decrease in equity]{withdrawals, expenses(cost of goods sold)}
\end{description}
Credits
\begin{description}
	\item[Increase in equity]{Capital, revenue}
	\item[Increase in liabilities]{Accounts payable, accumulated depreciation, borrowing}
\end{description}
\end{multicols*}
\end{document}