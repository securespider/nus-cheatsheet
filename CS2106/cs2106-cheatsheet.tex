\documentclass{article}
\usepackage[a4paper, margin=3mm, landscape]{geometry}
\usepackage{multicol}
\usepackage{xcolor}
\usepackage{enumitem}
\usepackage{amsmath}
\usepackage{amsfonts}
\usepackage{listings}
\usepackage{soul}
\usepackage{graphicx}

\pdfinfo{
    /Title (ACC1701x.pdf)
    /Creator (TeX)
    /Producer (pdfTeX 1.40.0)
    /Author (Vincent Pang)
    /Subject (CS2106)
    /Keywords (CS2106, nus, cheatsheet, pdf)
}

\graphicspath{ {./img/} }

\pagestyle{empty}
\setcounter{secnumdepth}{0}
\setlength{\columnseprule}{0.25pt}

% Redefine section commands to use less space
\makeatletter
\renewcommand{\section}{\@startsection{section}{1}{0mm}%
    {-1ex plus -.5ex minus -.2ex}%
    {0.5ex plus .2ex}%x
{\normalfont\large\bfseries}}
\renewcommand{\subsection}{\@startsection{subsection}{2}{0mm}%
    {-1explus -.5ex minus -.2ex}%
    {0.5ex plus .2ex}%
{\normalfont\normalsize\bfseries}}
\renewcommand{\subsubsection}{\@startsection{subsubsection}{3}{0mm}%
    {-1ex plus -.5ex minus -.2ex}%
    {1ex plus .2ex}%
{\normalfont\small\bfseries}}%
\makeatother

% Adjust spacing for all itemize/enumerate
\setlength{\leftmargini}{0.5cm}
\setlength{\leftmarginii}{0.5cm}
\setlist[itemize,1]{leftmargin=2mm,labelindent=1mm,labelsep=1mm}
\setlist[itemize,2]{leftmargin=2mm,labelindent=1mm,labelsep=1mm}

% Font
\renewcommand{\familydefault}{\sfdefault}

% Define colors for math formulas
\definecolor{myblue}{cmyk}{1,.72,0,.38}
\everymath\expandafter{\the\everymath \color{myblue}}

% Custom command for keywords
\definecolor{highlight}{RGB}{251,243,218}
\newcommand{\keyword}[2][]{\sethlcolor{highlight}\hl{\textbf{#2}} #1 - }
\newcommand{\ilkeyword}[1]{\sethlcolor{highlight}\hl{\textbf{#1}}}

% Define colors and style for code
\definecolor{codegreen}{rgb}{0,0.6,0}
\definecolor{codegray}{rgb}{0.5,0.5,0.5}
\definecolor{codered}{HTML}{CC241D}
\definecolor{backcolor}{rgb}{0.95,0.95,0.95}
\lstdefinestyle{codestyle}{
    backgroundcolor = \color{backcolor},
    commentstyle = \color{codegray},
    keywordstyle = \color{codered},
    stringstyle = \color{codegreen},
    basicstyle = \ttfamily,
    breakatwhitespace = false,
    showstringspaces = false,
    breaklines = true,
    showtabs = false,
    tabsize = 2
}
\lstset{style = codestyle}

% -----------------------------------------------------------------------
\begin{document}
\begin{multicols*}{3}
\footnotesize

% Title box
\begin{center}
    \fbox{
        \parbox{0.8\linewidth}{
            \centering \textcolor{black}{
                {\Large\textbf{CS2106}} \\
                \normalsize{AY22/23 Sem 2}} \\
                {\footnotesize \textcolor{gray}{github.com/securespider}}
        }
    }
\end{center}
\section{01. Introduction}
\keyword{OS}{Program that acts as an intermediary between user and hardware}

\subsection{Different architectures}
\subsubsection{Harvard architecture}
\includegraphics[scale=0.2]{harvard-architecture}
\subsubsection{Von Neumann architecture}
\includegraphics[scale=0.2]{von-neumann}

\begin{description}
	\item[Difference]{Separate vs common storage pathway for code and data}
\end{description}
Why do we need OS?

\subsection{Mainframe}
Old analog "computers" using physical cards for programming
\subsubsection{Improvements}
\begin{itemize}
	\item Problem: Batch processing inefficient
	\item Solution: Multiprogramming
	\begin{itemize}
		\item Loading multiple jobs that runs while other jobs using I/O
		\item Overlapping computation with I/O
	\end{itemize}
	\item Problem: Only one user 
	\item Solution: Time sharing OS
	\begin{itemize}
		\item Multiple concurrent users using terminals
		\item User job scheduling 
		\item Memory management
		\item \keyword{Hardware virtualization}{Each program executes as if it had all resources}
	\end{itemize}
\end{itemize}


\subsection{Motivation}
\begin{enumerate}
	\item Abstraction
	\begin{itemize}
		\item Hide low level details and present common, high-level functionality to users
	\end{itemize}
	\item Resource allocation
	\begin{itemize}
		\item Allow concurrent usage of resource and execute programs simultaneously
		\item Arbitrate conflicting request fairly and efficiently
	\end{itemize}
	\item Control programs
	\begin{itemize}
		\item Restrict resource allocation
		\item Security, protection and error prevention
		\item Ensure proper use of device
	\end{itemize}
\end{enumerate}
\subsubsection{Advantage}
\begin{itemize}
	\item Portable and flexible
	\item Use computer resources efficiently
\end{itemize}
\subsubsection{Disadvantage}
\begin{itemize}
	\item Significant overhead
\end{itemize}

\subsubsection{OS vs User Program}
Similarities
\begin{itemize}
	\item Both softwares
\end{itemize}
Difference
\begin{itemize}
	\item OS runs in \keyword{kernel mode}{Access to all hardware resources}
	\item User programs run in \keyword{User mode}{Limited access}
	\item User programs use syscalls to communicate with OS for hardware processes
\end{itemize}
\includegraphics[scale=0.2]{os-interaction}
\\Why OS dont occupy entire hardware layer
\begin{itemize}
	\item Slow to have all operations pass through intermediary
	\item User programs can have direct interaction with hardware (eg. Arithmetic) during low risk operations
\end{itemize}

\subsection{OS structure}
\subsubsection{Monolithic OS}
\begin{itemize}
	\item One big kernel program
	\item Well understood and has good performance
	\item Highly \keyword{coupled}{internal structure interconnected that unintentionally affect each other}
\end{itemize}
\includegraphics[scale=0.2]{monolithic-os}

\subsubsection{Microkernel}
\begin{itemize}
	\item Small clean
	\item Basic and essential facilities
	\item IPC communication OR run external programs outside OS
	\item Robust and more \keyword{modular}{Extendible and maintainable}
	\item Better isolation btw kernel and services
	\item Lower performance
\end{itemize}
\includegraphics[scale=0.32]{microkernel-os}
%--------------------------------------------------------------------------------------------------------------------

\section{02. Process abstraction}
\subsection{Motivation}
\begin{itemize}
	\item Allow concurrent usage of hardware
	\item Multiple programs sharing the same processors/IO
\end{itemize}

\subsection{Computer organisation}
\includegraphics[scale=0.27]{computer-organisation}
\subsubsection{Memory}
\begin{itemize}
	\item Storage for instruction and data
	\item Managed by the OS
	\item Normally accessed via load/store instructions
\end{itemize}
\subsubsection{Cache}
\begin{itemize}
	\item Fast and invisible to software
	\item Duplicate part of the memory for faster access
	\item Usually split into instruction and data cache
\end{itemize}
\subsubsection{Fetch}
\begin{itemize}
	\item Load instructions from memory
	\item Location indicated by \textbf{Program Counter}
\end{itemize}
\subsubsection{Functional units}
\begin{itemize}
	\item Carry out instruction execution
	\item Dedicated to specific instruction type
\end{itemize}
\subsubsection{Registers}
\begin{itemize}
	\item Internal storage for fastest access speed
\end{itemize}

\subsection{Information needed}
\begin{itemize}
	\item Memory context
	\begin{itemize}
		\item Code
		\item Data
	\end{itemize}
	\item Hardware context
	\begin{itemize}
		\item Register
		\item PC value
		\item Frame Pointer
	\end{itemize}
	\item OS context
	\begin{itemize}
		\item Process properties
		\item Resources used
		\item Files
	\end{itemize}
\end{itemize}

\subsection{Function calls}
\subsubsection{Separation of text and data}
Suppose a function f() calls g()
\begin{itemize}
	\item f is caller and g is callee
\end{itemize}
Steps of control flow
\begin{enumerate}
	\item Setup parameters
	\item Trf ctrl to callee
	\item Setup local var
	\item Store any results
	\item Return ctrl to caller
\end{enumerate}
\subsubsection{Issues}
Control Flow
\begin{itemize}
	\item Need to jump to functional body when callee called
	\item Need to resume to next instruction in caller after done
\end{itemize}
Data storage
\begin{itemize}
	\item Need to pass parameters to function
	\item Need to capture return result
	\item May have local variables
\end{itemize}
Additional
\begin{itemize}
	\item May lead to overriding of data in caller by callee (interference)
	\item Calling g() multiple times may lead to insufficient space and overriding
\end{itemize}


\subsection{Stack memory}
Memory to store function invocation
\keyword{Stack Pointer}{Indicates the first free location in the stack region}\\
\keyword{Frame Pointer}{Points to the frame and is used for traversing around the stack easily}\\
\includegraphics[scale=0.13]{stack-frame}\\
Information needed for function invocation - Stack frame
\begin{itemize}
	\item Return address of caller
	\item Arguments for the function
	\item Local variables
	\item Stack and frame pointer of caller
	\item GPR values (register spilling)
\end{itemize}
Callee stack frame will be on top of the caller


\subsection{Dynamic memory (Heap)}
Memory that the program/user specifies manually (eg. malloc, new)\\
Problems:
\begin{itemize}
	\item Allocated only at runtime
	\begin{itemize}
		\item Size not known at program compilation time
		\item Cannot specify a region in data 
	\end{itemize}
	\item No definite dellocation timing
	\begin{itemize}
		\item Must be freed explicitly by the program
		\item Cannot place in stack region
	\end{itemize}
\end{itemize}
Solution:\\
Add a region "Heap for dynamic allocation\\
Problems with heap memory:
\begin{itemize}
	\item Generation of holes in between data due to variable deallocation timing
\end{itemize}

%-----------------------------------------------------------------------------------------------------------------------
\subsection{OS context}
\subsubsection{Process identification}
Features:
\begin{itemize}
	\item Distinguish processes from each other (Unique)
	\item Communicated to the hardware
\end{itemize}
\subsubsection{Process state}
\begin{itemize}
	\item Denotes whether a process is running or not (Running vs waiting vs not running)
\end{itemize}
\includegraphics[scale=0.48]{process-model}

\keyword{Process control block}{Table representing all processes}

\subsection{Exceptions and interrupts}
Exceptions
\begin{itemize}
	\item Synchronous (due to program execution)
	\item Machine level instructions arise errors
	\item Exception handler executed automatically in software
\end{itemize}
Interrupts
\begin{itemize}
	\item Asynchronous (Can happen anytime)
	\item External events that cause execution to fail (hardware related errors)
	\item Program execution suspended and \textbf{interrupt handler} executed automatically
\end{itemize}

\subsubsection{Instruction execution}
\begin{enumerate}
	\item Read byte from PC and decode instruction
	\item Read 2 bytes to get the address/operands
	\item Perform ALU operations
	\item Store result into destination
	\item Check if any interruptions
\end{enumerate}
Interrupts can happen at anytime, and will remain pending until step 5 where it is handled

\subsubsection{Interruption handling}
\begin{enumerate}
	\item Push PC and status register into hardware
	\item Disable interrupts
	\item Read \keyword{Interrupt Vector Table}{Table where the OS stores address of all interrupt handlers}
	\item Switch to kernel mode
	\item Set PC to handler address and execute the instructions
\end{enumerate}
\begin{itemize}
	\item OS populates the IVT table with address of interrupt routines
	\item Hardware reads IVT to locate the handler
\end{itemize}

\subsection{System calls}
\keyword{Application Program Interface}{Provides way of calling facilities/services in kernel}\\
Instructions can only be done in kernel mode

\subsubsection{Method}
\begin{itemize}
	\item Library version with the same name and same arguments
	\item User friendly library version
	\item using the function $long~syscall(long~number);$
\end{itemize}
\subsubsection{Mechanism}
\begin{enumerate}
	\item User invoke library call
	\item Place call number in the designated location
	\item Library call executes a special instruction (\textbf{TRAP/syscall}) to change user to kernel mode
	\item (in kernel) syscall handler is determined (by a \textbf{dispatcher})
	\item syscall handler is executed
	\item Syscall handler ends and control returned to the library call
	\item Return to user mode and continue normal function mechanism
\end{enumerate}



%--------------------------------------------------------------------------------------------------------------------
\section{03. Process abstraction in Unix}
Process information
\begin{description}
	\item[Pid]
	\item[Process state]{Running, sleeping, stopped, \keyword{zombie}{Process that has stopped but resources not cleared}}
	\item[Parent pid]
	\item[Cumulative CPU time]{For scheduling}
\end{description}

\subsection{fork()}
Process creation
\begin{description}
	\item[Package]{unistd.h and sys/types.h}
	\item[return]{PID of newly created process(parent) and 0 (child process)}
\end{description}
\subsubsection{Behaviour}
\begin{itemize}
	\item Creates a child process
	\begin{itemize}
		\item \textbf{Copy} data of parent (Independent memory space)
		\item Sane code, same address space
		\item Differs by pid, ppid and fork() return value
	\end{itemize}
\end{itemize}

\subsubsection{Implementation}
Clone the parent process 
\begin{enumerate}
	\item Create address space of child process
	\item allocate new pid to child and pass to parent
	\item Create kernel process data structures
	\item Copy kernel environment of parent process
	\item Initialize child process context (pid, ppid, cpu_time = 0)
	\item Copy memory regions from parent
	\begin{itemize}
		\item Very expensive operation
		\item Code, data, stack
	\end{itemize}
	\item Acquire shared resources
	\item Initialize hardware context for child process (copy parent registers)
\end{enumerate}

Problem: Memcopy is very expensive operation\\
Solution: Copy on write
\begin{itemize}
	\item Only duplicate a memory location when it is written to
\end{itemize}

\subsection{exec()}
Replace current executing process image
\begin{itemize}
	\item code and data replaced
	\item PID intact
\end{itemize}
Format
\begin{description}
	\item[Param]{char *path, char *arg0...}
	\begin{itemize}
		\item Note that last term MUST be \textbf{NULL} indicating end of argument list
	\end{itemize}
	\item[header]{unistd.h}
\end{description}

\subsection{exit()}
\begin{description}
	\item[return]{Does not return anything}
	\item[param]{Status to be returned to the wait call}
\end{description}
\begin{itemize}
	\item Most system resource used by process are released on exit
	\item return from main() implicitly calls exit(0)
	\item Basic processes are not releasable
	\begin{itemize}
		\item Pid and status
		\item Process accounting info
	\end{itemize}
\end{itemize}

\subsection{wait()}
Parent child synchronisation
\begin{description}
	\item[param]{&status - address to put return value}
	\item[header]{sys/types.h and sys/wait.h}
	\item[return]{pid of terminated process}
\end{description}
\begin{itemize}
	\item Call is blocking - suspend operation until at least one child terminates
	\item Cleans up remainder of child system resources (PID, status)
	\item waitpid - used for waiting for specific child process
\end{itemize}

\subsection{Orphan and zombie process}
\keyword{zombie}{Process that has exited but parent did not call wait}\\
\keyword{Orphan}{Child process whose parent has been terminated}
\begin{itemize}
	\item Parenthood will be propagated up to init which may use wait() to clean up
\end{itemize}


%--------------------------------------------------------------------------------------------------------------------
\section{04. Inter Process Communication}
%--------------------------------------------------------------------------------------------------------------------
%--------------------------------------------------------------------------------------------------------------------
\end{multicols*}
\end{document}
