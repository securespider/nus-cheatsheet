\documentclass{article}
\linespread{0.7}
\usepackage[a4paper, margin=3mm, landscape]{geometry}
\usepackage{multicol}
\usepackage{xcolor}
\usepackage{enumitem}
\usepackage{amsmath}
\usepackage{amsfonts}
\usepackage{listings}
\usepackage{soul}
\usepackage{graphicx}

\pdfinfo{
    /Title (ee3801-cheatsheet.pdf)
    /Creator (TeX)
    /Producer (pdfTeX 1.40.0)
    /Author (Vincent Pang)
    /Subject (template)
    /Keywords (cheatsheet, pdf, ee3801)
}

\graphicspath{ {./img/} }

\pagestyle{empty}
\setcounter{secnumdepth}{0}
\setlength{\columnseprule}{0.25pt}

% Redefine section commands to use less space
\makeatletter
\renewcommand{\section}{\@startsection{section}{1}{0mm}%
    {-1ex plus -.5ex minus -.2ex}%
    {0.5ex plus .2ex}%x
{\normalfont\large\bfseries}}
\renewcommand{\subsection}{\@startsection{subsection}{2}{0mm}%
    {-1explus -.5ex minus -.2ex}%
    {0.5ex plus .2ex}%
{\normalfont\normalsize\bfseries}}
\renewcommand{\subsubsection}{\@startsection{subsubsection}{3}{0mm}%
    {-1ex plus -.5ex minus -.2ex}%
    {1ex plus .2ex}%
{\normalfont\small\bfseries}}%
\makeatother

% Adjust spacing for all itemize/enumerate
\setlength{\leftmargini}{0.5cm}
\setlength{\leftmarginii}{0.5cm}
\setlist[itemize,1]{leftmargin=2mm,labelindent=1mm,labelsep=1mm}
\setlist[itemize,2]{leftmargin=2mm,labelindent=1mm,labelsep=1mm}

% Font
\renewcommand{\familydefault}{\sfdefault}

% Define colors for math formulas
\definecolor{myblue}{cmyk}{1,.72,0,.38}
\everymath\expandafter{\the\everymath \color{myblue}}

% Custom command for keywords
\definecolor{highlight}{RGB}{251,243,218}
\newcommand{\keyword}[2][]{\sethlcolor{highlight}\hl{\textbf{#2}} #1 - }
\newcommand{\ilkeyword}[1]{\sethlcolor{highlight}\hl{\textbf{#1}}}

% Define colors and style for code
\definecolor{codegreen}{rgb}{0,0.6,0}
\definecolor{codegray}{rgb}{0.5,0.5,0.5}
\definecolor{codered}{HTML}{CC241D}
\definecolor{backcolor}{rgb}{0.95,0.95,0.95}
\lstdefinestyle{codestyle}{
    backgroundcolor = \color{backcolor},
    commentstyle = \color{codegray},
    keywordstyle = \color{codered},
    stringstyle = \color{codegreen},
    basicstyle = \ttfamily,
    breakatwhitespace = false,
    showstringspaces = false,
    breaklines = true,
    showtabs = false,
    tabsize = 2
}
\lstset{style = codestyle}

% -----------------------------------------------------------------------
\begin{document}
\begin{multicols*}{3}
\footnotesize

% Title box
\begin{center}
    \fbox{
        \parbox{0.8\linewidth}{
            \centering \textcolor{black}{
                {\Large\textbf{EE3801 Cheatsheet}} \\
                \normalsize{Intro to Data Engineering}} \\
                {\footnotesize \textcolor{gray}{github.com/securespider}}
        }
    }
\end{center}
\section{01.1 Intro}
\subsection{Data science vs engineering}
\begin{itemize}
	\item \keyword{Science}{Learn, optimise, analytics, aggregate and labelling}
	\item \keyword{Engineering}{Cleaning, data storage, logging, sensors, pipelines}
\end{itemize}
\subsection{Data structure}
\subsubsection{Unstructured data}
\begin{itemize}
	\item{Chaotic no order to data}
\end{itemize}
\subsubsection{Structured data}
\begin{itemize}
	\item Data stored access in the same format
\end{itemize}
\subsubsection{Semi structured data}
\begin{itemize}
	\item Can contain both forms of data 
	\item Some structure but not all data points follow same format
\end{itemize}
\subsection{Big data}
\begin{description}
	\item[Volume, Variety, Variability]
	\item[Velocity]{High rate of data generation}
	\begin{itemize}
		\item Must create a robust and scalable pipeline
	\end{itemize}
\end{description}
\subsection{Raw Data}
\begin{itemize}
	\item Tend to have gaps
\end{itemize}
\subsubsection{Data wrangling}
Used to understand raw data
\begin{description}
	\item[Discovery]{Understand what is in your data}
	\item[Structure]
	\item[Cleaning]{Dealing with gaps (nulls), outliers, formatting bugs}
	\item[Enrichment]{Derive other data from other information/ additional data augmentation (feature selection)}
	\item[Validation]{Verify data quality, sources}
	\item[Publishing]{Give data scientist}
\end{description}
\subsection{Process}
\begin{description}
	\item[Extraction]{Retrieve raw data from unstructured pool and migrate to temp repo}
	\item[Transformation]{Structure enrich and convert raw data}
	\item[Loading]{Loading structured data into data warehouse}
\end{description}
\subsubsection{Data warehouse}
Decision support system storing historical data from organisations
\subsubsection{Data Pipeline}
\begin{itemize}
	\item Processing underlying raw data in ordered sequence of steps
\end{itemize}

\section{01.2 Data Pipelines}
\subsection{Considerations}
\subsubsection{Big data}
\begin{description}
	\item[Velocity]{Streaming, captured and processed in real time}
	\item[Volume]{Scalable wrt time}
	\item[Variety]{Recognise and process diff formats}
\end{description}
\subsubsection{Business}
\begin{itemize}
	\item Handling streaming data?
	\item How much data to expect (Time horizon/how much storage consumed)
	\item What type/how much processing in DP
	\item Where is data source? Need micro-services?
\end{itemize}
\subsection{Architecture}
\subsubsection{Batch-based DP}
\begin{itemize}
	\item Analysis of data that has been stored over a period of time
	\item $N$ independent tasks to process with $k$ stages
	\item Each stage takes max of $T$ time process input
	\item Diff stage can operate concurrently
	\item $t(N,k) = T\times(N+k-1)$
\end{itemize}
\subsubsection{Streaming-based DP}
\begin{itemize}
	\item Processing as data flows through system
	\item Logging and persistent result storage
\end{itemize}
\subsubsection{Lambda Architecture}
\begin{itemize}
	\item Combination of batch and streaming
	\item Separate processing engine for "batch" and "speed" layers combining in "service" layer
	\item Accounts for real-time streaming and historical batch analysis
	\item Encourage raw data storage and create new dst for queries
	\item Min errors for both layers reliably at fast speeds
\end{itemize}
\subsubsection{Kappa Architecture}
\begin{itemize}
	\item Replay data and process both layers in same single stream processing engine
	\item Good for big data architecture with cheaper hardware and focus on stream
\end{itemize}
\subsection{Design}
\begin{enumerate}
	\item Identify application and decide if DP needed
	\item Identify DP category (architecture)
	\item Understand working mechanism, parameters/variables
\end{enumerate}
\end{multicols*}
\end{document}
