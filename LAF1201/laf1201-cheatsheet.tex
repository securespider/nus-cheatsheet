\documentclass{article}
\linespread{0.7}
\usepackage[a4paper, margin=3mm, landscape]{geometry}
\usepackage{multicol}
\usepackage{xcolor}
\usepackage{enumitem}
\usepackage{amsmath}
\usepackage{amsfonts}
\usepackage{listings}
\usepackage{soul}
\usepackage{graphicx}

\pdfinfo{
    /Title (template.pdf)
    /Creator (TeX)
    /Producer (pdfTeX 1.40.0)
    /Author (securespider)
    /Subject (template)
    /Keywords (cheatsheet, pdf)
}

\graphicspath{ {./img/} }

\pagestyle{empty}
\setcounter{secnumdepth}{0}
\setlength{\columnseprule}{0.25pt}

% Redefine section commands to use less space
\makeatletter
\renewcommand{\section}{\@startsection{section}{1}{0mm}%
    {-1ex plus -.5ex minus -.2ex}%
    {0.5ex plus .2ex}%x
{\normalfont\large\bfseries}}
\renewcommand{\subsection}{\@startsection{subsection}{2}{0mm}%
    {-1explus -.5ex minus -.2ex}%
    {0.5ex plus .2ex}%
{\normalfont\normalsize\bfseries}}
\renewcommand{\subsubsection}{\@startsection{subsubsection}{3}{0mm}%
    {-1ex plus -.5ex minus -.2ex}%
    {1ex plus .2ex}%
{\normalfont\small\bfseries}}%
\makeatother

% Adjust spacing for all itemize/enumerate
\setlength{\leftmargini}{0.5cm}
\setlength{\leftmarginii}{0.5cm}
\setlist[itemize,1]{leftmargin=2mm,labelindent=1mm,labelsep=1mm}
\setlist[itemize,2]{leftmargin=2mm,labelindent=1mm,labelsep=1mm}

% Font
\renewcommand{\familydefault}{\sfdefault}

% Define colors for math formulas
\definecolor{myblue}{cmyk}{1,.72,0,.38}
\everymath\expandafter{\the\everymath \color{myblue}}

% Custom command for keywords
\definecolor{highlight}{RGB}{251,243,218}
\newcommand{\keyword}[2][]{\sethlcolor{highlight}\hl{\textbf{#2}} #1 - }
\newcommand{\ilkeyword}[1]{\sethlcolor{highlight}\hl{\textbf{#1}}}

% Define colors and style for code
\definecolor{codegreen}{rgb}{0,0.6,0}
\definecolor{codegray}{rgb}{0.5,0.5,0.5}
\definecolor{codered}{HTML}{CC241D}
\definecolor{backcolor}{rgb}{0.95,0.95,0.95}
\lstdefinestyle{codestyle}{
    backgroundcolor = \color{backcolor},
    commentstyle = \color{codegray},
    keywordstyle = \color{codered},
    stringstyle = \color{codegreen},
    basicstyle = \ttfamily,
    breakatwhitespace = false,
    showstringspaces = false,
    breaklines = true,
    showtabs = false,
    tabsize = 2
}
\lstset{style = codestyle}

% -----------------------------------------------------------------------
\begin{document}
\begin{multicols*}{5}
\footnotesize

% Title box
\begin{center}
    \fbox{
        \parbox{0.8\linewidth}{
            \centering \textcolor{black}{
                {\Large\textbf{Subject code}} \\
                \normalsize{[subject]}} \\
                {\footnotesize \textcolor{gray}{github.com/securespider}}
        }
    }
\end{center}
\section{01.1 Introduction}
\subsection{Introduce}
\begin{description}
	\item[Comment tu t'appelles]{What you call yourself}
	\item[Je m'appelle xx]{My name is xx}
	\item[vous]{each person/formal vers}
	\item[appelle/appelles/appellez/appelons/appellent]
\end{description}
\subsection{English-french words}
\begin{description}
	\item[petite]{small}
	\item[souvenir]{memory/souvenir}
	\item[barrage]{dam}
	\item[rendezvous]{appointment}
\end{description}
\subsection{TB}
\begin{description}
	\item[Bonjour]{good day}
	\item[Bonsoir]{good evening}
	\item[femme]{woman}
	\item[fille]{girl}
	\item[homme]{man}
	\item[garçon]{boy}
	\item[c'est]{it is}
	\item[elle/il]{she/him}
	\item[nous/elles/ils]{we/they(girl)/they(guy+girl)}
	\item[être/je suis/tu es/il est]{is/are}
	\item[nous sommes/vous êtes/ils sont]
	\item[au revoir]{bye}
	\item[salut]{hello}
	\item[province]{france w/o paris}
	\item[connais]{know/been to}
	\item[matin/midi/'apres midi/soir/nuit]{morning/noon/afternoon/evening/night}
	\item[écoutez et répondez]{listen and respond}
	\item[parle/parles/parlez/parlons/parlent]{speak}
	\item[lisez]{read}
	\item[écrivez]{write}
	\item[voisine/voisin]{neighbour(vuazin/voizon)}
\end{description}
\subsubsection{Day of week}
\begin{itemize}
	\item lundi/mardi/mercredi/jeudi/vendredi	
	\item samedi/dimarche
\end{itemize}
\subsubsection{Numbers}
\begin{description}
	\item[un/deux/trois/quatre/cinq]{1/2/3/4/5}
	\item[six/sept/huit/neuf/dix]{6/7/8/9/10}
	\item[plus/moins/égaler]{plus/minus/equals}
\end{description}
\subsubsection{Questions}
\begin{description}
	\item[qui]{who}
	\item[que/quoi]{what}
	\item[comment]{how}
	\item[ou]{where}
	\item[quand]{when}
	\item[pourquoi]{why}
	\item[combien]{how much}
	\item[quel/quels/quelle/quelles]{who/which}
\end{description}
\section{01.2 C'est qui}
\begin{description}
	\item[Elle, elle s'appelle xx]{Her, she calls herself xx}
	\item[Lui, il "]{Him, he "}
	\begin{itemize}
		\item Context: ordering for others
	\end{itemize}
	\item[encore]{again}
\end{description}
\subsubsection{Colours}
\begin{description}
	\item[blanc/blanche]{white}
	\item[vert]{green}
	\item[noir]{black}
	\item[bleu]{blue}
	\item[rouge]{red}
	\item[jaune]{yellow}
	\item[gris]{gray}
	\item[violet]{purple}
	\item[marron]{brown}
	\item[orange]{orange}
\end{description}
\begin{itemize}
	\item When used as nouns is masculine
	\item in white = en blanc
\end{itemize}
\subsubsection{Occupation}
\begin{description}
	\item[Chanteur/Chanteuse]{Singer}
	\item[Nageur/"]{Swimmer}
	\item[acteur/"]{Actor}
\end{description}
\begin{itemize}
	\item -euse = woman -eur = man
\end{itemize}
\subsubsection{Nouns}
\begin{description}
	\item[un sac]{bag}
	\item[une montre]{watch}
	\item[un téléphone]{telephone}
	\item[une maison]{house}
\end{description}
\subsubsection{Tonique}
\begin{itemize}
	\item moi/toi/elle/lui
	\item nous/vous/elles/eux
\end{itemize}
\subsubsection{Verbs}
\begin{description}
	\item[-er]{e/es/ez/ons/ent}
\end{description}
\subsubsection{Pronunciation}
\begin{description}
	\item[tiret]{-}
\end{description}
\section{02.1}
\subsubsection{Definite articles}
\begin{description}
	\item[le]{eg. Canada}
	\begin{itemize}
		\item féminin eg. table, maison
		\item \tiny{Mexique, Cambodge, Zimbabwé, Mozambique}
	\end{itemize}
	\item[les]{Seychell\hl{es}}
	\begin{itemize}
		\item Plural les hotels
	\end{itemize}
	\item[la]{eg. Franc\hl{e}}
	\begin{itemize}
		\item masuline 
	\end{itemize}
	\item[l']{\hl{A}lgéria}
	\begin{itemize}
		\item Start vowel
	\end{itemize}
\end{description}
\section{02.2}
\subsubsection{Numbers}
\begin{description}
	\item[vingt]{20}
	\item[trente]{30}
	\item[quarante]{40}
	\item[cinquante]{50}
	\item[soixante]{60}
\end{description}
\subsection{TB}
\begin{description}
	\item[avoir | j'ai | tu as]{have}
	\item[il/elle a | nous avons]
	\item[vous avez | ils/elles ont]
	\item[Ils ont quel âge?]{What his age}
	\item[Quel âge vous avez]{What is your age}
	\item[j'ai vingt ans]{i have 20 years}
	\item[aussi]{also}
	\item[un peu]{a little}
	\item[je suis triste]{sad}
	\item[j'ai soif]{thirst}
	\item[je suis fatigué]{tired}
	\item[j'ai chaud]{hot}
	\item[j'ai froid]{cold}
	\item[je suis peur]{shocked}
	\item[j'ai fiem]{hungry}
\end{description}
\begin{itemize}
	\item Adjective takes the gender/val of noun
\end{itemize}
\end{multicols*}
\end{document}



