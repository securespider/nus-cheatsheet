\documentclass{article}
\linespread{0.7}
\usepackage[a4paper, margin=3mm, landscape]{geometry}
\usepackage{multicol}
\usepackage{xcolor}
\usepackage{enumitem}
\usepackage{amsmath}
\usepackage{amsfonts}
\usepackage{listings}
\usepackage{soul}
\usepackage{graphicx}

\pdfinfo{
    /Title (template.pdf)
    /Creator (TeX)
    /Producer (pdfTeX 1.40.0)
    /Author (securespider)
    /Subject (template)
    /Keywords (cheatsheet, pdf)
}

\graphicspath{ {./img/} }

\pagestyle{empty}
\setcounter{secnumdepth}{0}
\setlength{\columnseprule}{0.25pt}

% Redefine section commands to use less space
\makeatletter
\renewcommand{\section}{\@startsection{section}{1}{0mm}%
    {-1ex plus -.5ex minus -.2ex}%
    {0.5ex plus .2ex}%x
{\normalfont\large\bfseries}}
\renewcommand{\subsection}{\@startsection{subsection}{2}{0mm}%
    {-1explus -.5ex minus -.2ex}%
    {0.5ex plus .2ex}%
{\normalfont\normalsize\bfseries}}
\renewcommand{\subsubsection}{\@startsection{subsubsection}{3}{0mm}%
    {-1ex plus -.5ex minus -.2ex}%
    {1ex plus .2ex}%
{\normalfont\small\bfseries}}%
\makeatother

% Adjust spacing for all itemize/enumerate
\setlength{\leftmargini}{0.5cm}
\setlength{\leftmarginii}{0.5cm}
\setlist[itemize,1]{leftmargin=2mm,labelindent=1mm,labelsep=1mm}
\setlist[itemize,2]{leftmargin=2mm,labelindent=1mm,labelsep=1mm}

% Font
\renewcommand{\familydefault}{\sfdefault}

% Define colors for math formulas
\definecolor{myblue}{cmyk}{1,.72,0,.38}
\everymath\expandafter{\the\everymath \color{myblue}}

% Custom command for keywords
\definecolor{highlight}{RGB}{251,243,218}
\newcommand{\keyword}[2][]{\sethlcolor{highlight}\hl{\textbf{#2}} #1 - }
\newcommand{\ilkeyword}[1]{\sethlcolor{highlight}\hl{\textbf{#1}}}

% Define colors and style for code
\definecolor{codegreen}{rgb}{0,0.6,0}
\definecolor{codegray}{rgb}{0.5,0.5,0.5}
\definecolor{codered}{HTML}{CC241D}
\definecolor{backcolor}{rgb}{0.95,0.95,0.95}
\lstdefinestyle{codestyle}{
    backgroundcolor = \color{backcolor},
    commentstyle = \color{codegray},
    keywordstyle = \color{codered},
    stringstyle = \color{codegreen},
    basicstyle = \ttfamily,
    breakatwhitespace = false,
    showstringspaces = false,
    breaklines = true,
    showtabs = false,
    tabsize = 2
}
\lstset{style = codestyle}

% -----------------------------------------------------------------------
\begin{document}
\begin{multicols*}{5}
\footnotesize

% Title box
\begin{center}
    \fbox{
        \parbox{0.8\linewidth}{
            \centering \textcolor{black}{
                {\Large\textbf{Subject code}} \\
                \normalsize{[subject]}} \\
                {\footnotesize \textcolor{gray}{github.com/securespider}}
        }
    }
\end{center}
\section{01.1 Introduction}
\subsection{Introduce}
\begin{description}
	\item[Comment tu t'appelles]{What you call yourself}
	\item[Je m'appelle xx]{My name is xx}
	\item[vous]{each person/formal vers}
	\item[appeler/je/tu/il/vous/nous/ils]{call/-e/-es/-e/-ez/-ons/-ent}
\end{description}
\subsection{English-french words}
\begin{description}
	\item[petite]{small}
	\item[souvenir]{memory/souvenir}
	\item[barrage]{dam}
	\item[rendezvous]{appointment}
\end{description}
\subsection{TB}
\begin{description}
	\item[Bonjour]{good day}
	\item[Bonsoir]{good evening}
	\item[femme]{woman}
	\item[petite fille]{girl}
	\item[homme]{man}
	\item[garçon]{boy}
	\item[c'est]{it is}
	\item[elle/il]{she/him}
	\item[nous/elles/ils]{we/they(girl)/they(guy+girl)}
	\item[être/je suis/tu es/il est]{is/are}
	\item[nous sommes/vous êtes/ils sont]
	\item[au revoir]{bye}
	\item[salut]{hello}
	\item[voyelles]{vowels}
	\item[consonnes]{consonants}
	\item[serrer]{shake}
	\item[la main]{hand}
	\item[bise]{kiss greeting}
	\item[pause]{pause}
	\item[marche]{continue}
	\item[suivez]{follow}
	\item[province]{france w/o paris}
	\item[connais]{know/been to}
	\item[matin/midi/'apres midi/soir/nuit]{morning/noon/afternoon/evening/night}
	\item[écoutez et répondez]{listen and respond}
	\item[parle/parles/parlez/parlons/parlent]{speak}
	\item[lisez]{read}
	\item[écrivez]{write}
	\item[voisine/voisin]{neighbour(vuazin/voizon)}
\end{description}
\subsubsection{Day of week}
\begin{itemize}
	\item lundi/mardi/mercredi/jeudi/vendredi	
	\item samedi/dimarche
\end{itemize}
\subsubsection{Numbers}
\begin{description}
	\item[un/deux/trois/quatre/cinq]{1/2/3/4/5}
	\item[six/sept/huit/neuf/dix]{6/7/8/9/10}
	\item[plus/moins/égaler]{plus/minus/equals}
\end{description}
\subsubsection{Questions}
\begin{description}
	\item[qui]{who}
	\item[que/quoi]{what}
	\item[comment]{how}
	\item[ou]{where}
	\item[quand]{when}
	\item[pourquoi]{why}
	\item[combien]{how much}
	\item[quel/quels/quelle/quelles]{who/which}
	\item[présenter]{to present}
\end{description}
\section{01.2 C'est qui}
\begin{description}
	\item[Elle, elle s'appelle xx]{Her, she calls herself xx}
	\item[Lui, il "]{Him, he "}
	\begin{itemize}
		\item Context: ordering for others
	\end{itemize}
	\item[encore]{again}
	\item[quelqu'un]{someone}
	\item[se]{yourself}
\end{description}
\subsubsection{La couleur (all masculine)}
\begin{description}
	\item[blanc/blanche]{white}
	\item[vert]{green}
	\item[noir]{black}
	\item[bleu]{blue}
	\item[rouge]{red}
	\item[jaune]{yellow}
	\item[gris]{gray}
	\item[violet]{purple}
	\item[marron]{brown}
	\item[orange]{orange}
\end{description}
\begin{itemize}
	\item When used as nouns is masculine
	\item in white = en blanc
\end{itemize}
\subsubsection{La Profession/Métier}
\begin{description}
	\item[Chanteur/Chanteuse]{Singer}
	\item[Nageur/nageuse]{Swimmer}
	\item[acteur/actrice]{Actor}
	\item[musée]{museum}
	\item[cuisinière]{chef}
	\item[styliste]{stylist}
	\item[chef/cheffe]{chief}
	\item[professeur/professeure]{professor}
	\item[étudiant/étudiante]{student}
	\item[autuer/autrice]{author}
	\item[avocat/avocate]{lawyer}
	\item[pâtissier]{pastry chef}
	\item[journaliste]{journalist}
\end{description}
\begin{itemize}
	\item -euse = woman -eur = man
\end{itemize}
\subsubsection{Nouns}
\begin{description}
	\item[un sac]{bag}
	\item[une montre]{watch}
	\item[un téléphone]{telephone}
	\item[une maison]{house}
\end{description}
\subsubsection{Tonique}
\begin{itemize}
	\item moi/toi/elle/lui
	\item nous/vous/elles/eux
\end{itemize}
\subsubsection{Verbs}
\begin{description}
	\item[-er]{e/es/ez/ons/ent}
	\item[épeler]{spell}
\end{description}
\subsubsection{Pronunciation}
\begin{description}
	\item[tiret]{-}
\end{description}
\begin{description}
	\item[là]{there}
	\item[carte]{map}
\end{description}
\section{02.1}
\subsubsection{Definite articles}
\begin{description}
	\item[le]{the (m)}
	\begin{itemize}
		\item masuline eg. téléphone, bureau
		\item \tiny{Mexique, Cambodge, Zimbabwé, Mozambique}
	\end{itemize}
	\item[la]{the (f)}
	\begin{itemize}
		\item féminin eg. table, maison
		\item Countries end -e eg. Franc\hl{e}
	\end{itemize}
	\item[les]{the (p)}
	\begin{itemize}
		\item Plural les élèves, les livres, les hôtels
		\item \tiny{Pays Bas, États-Unis, Philippines, Seychelles}
	\end{itemize}
	\item[l']{the (start w vowel)}
	\begin{itemize}
		\item Start vowel \hl{A}lgéria, l'\hl{h}ôtel
	\end{itemize}
\end{description}
\begin{itemize}
	\item Cities do not need definite articles
	\item eg. Singapour, Monaco
\end{itemize}
\subsubsection{Countries}
\begin{itemize}
	\item francais, anglais, japonais, pakistanais
	\item néo-zélandais polonais, singapourien
	\item australien, canadien, cambodgien malaisien
	\item vietnamien, indien, américain, belge
	\item chinois espagnol, allemand(german), russe
	\item suédois (swedish), suisse, grec (greek)
\end{itemize}
\subsubsection{TB}
\begin{description}
	\item[tu parles quelles langues]{speak what lang?}
	\item[choisir]{choose}
	\item[habite]{live}
	\item[ca va mal]{Horrible}
	\item[jeu de rôles]{roleplay}
\end{description}
\section{02.2}
\subsubsection{Numbers}
\begin{description}
	\item[11/12/13]{onze, douze, treize}
	\item[14/15/16]{quatorze/quinze/seize}
	\item[20/30/40]{vingt/trente/quarante}
	\item[50/60]{cinquante/soixante}
\end{description}
\subsection{TB}
\begin{description}
	\item[avoir/je/tu/il/vous/nous/ils]{have/ai/as/a/avez/avons/ont}
	\item[Ils ont quel âge?]{What are their age}
	\item[Quel âge vous avez]{What is your age}
	\item[j'ai vingt ans]{i have 20 years}
	\item[Quelle est sa nationalité]{what ur nationality}
	\item[mais]{but}
	\item[préférées]{favorite}
	\item[lecture]{(act of)reading}
	\item[aussi]{also}
	\item[un peu]{a little}
	\item[je suis content]{happy}
	\item[je suis triste]{sad}
	\item[je suis fatigué(e)]{tired}
	\item[je suis en forme]{Healthy}
	\item[je suis malade]{sick}
	\item[j'ai soif]{thirst}
	\item[j'ai chaud]{hot}
	\item[j'ai froid]{cold}
	\item[j'ai peur]{shocked}
	\item[j'ai faim]{hungry}
	\item[j'ai sommeil]{sleepy}
\end{description}
\begin{itemize}
	\item Adjective takes the gender/val of noun
\end{itemize}

\section{03.1 C'est qui}
\subsubsection{Questions}
\begin{description}
	\item[quelle est sa xx]{what is his/her xx}
	\item[encore une fois]{repeat one more time}
	\item[trouvez]{find}
	\item[quels numéros]{what numbers}
\end{description}
\subsubsection{Special e}
\begin{description}
	\item[e]{err}
	\item[é]{ayy-accent aigu}
	\item[e]{air-accent grave e}
	\item[ê]{air-accent circonflexe e}
\end{description}
\subsubsection{Country inhabitants}
\begin{description}
	\item[-ais/-ois/-ien]{Add to end}
	\item[né]{born in}
\end{description}
\subsubsection{Conjugations}
\begin{description}
	\item[Li\hl{re}]{read}
	\begin{itemize}
		\item je lis, tu lis, elle/il lit
		\item nous lisons, vous lisez, elles lisent
	\end{itemize}
	\item[ecri\hl{re}]{to write}
	\begin{itemize}
		\item Same concept, -s, -t, -ons, -ez, -ent
	\end{itemize}
\end{description}
\subsubsection{TB}
\begin{description}
	\item[chiffres]{figure}
	\item[puis]{then}
	\item[jeu(x)]{game}
	\item[physique]{physics}
	\item[chimie]{chemistry}
	\item[partenaire]{partner}
	\item[la culture]{culture}
	\item[la technologie]{technology}
	\item[l'amour]{love}
	\item[des sciences]{sciences}
	\item[des affaires]{politics}
	\item[l'humour]{comedy}
\end{description}

\section{03.2}
\begin{description}
	\item[s'exprimer]{express}
	\item[poliment]{politely}
	\item[remplir]{fill}
	\item[formulaire]{form}
	\item[le coûte]{price}
	\item[ce(m)/cette(f)]{this}
	\item[un sac]{bag}
	\item[une robe]{dress}
	\item[un café]{coffee}
	\item[mots]{words}
	\item[les mots de politesse]{polite words}
	\item[centimes]{cents}
	\item[eg. 1.50euro]{un euro cinquante}
	\item[c'est un xx]{its xx}
	\begin{itemize}
		\item Add un/une/des when noun
	\end{itemize}
	\item[dit]{say}
	\item[entendez]{heard}
	\item[retrouvez]{find}
	\item[désolé]{sorry}
	\item[je vous en prie]{ur welcome (polite)}
	\item[de rien]{your welcome}
	\item[courriel]{email}
	\item[mot de passe]{password}
	\item[civilité]{gender}
	\item[repérer]{spot}
	\item[aide]{help}
	\item[comprendre]{to understand}
	\item[souligner]{underline/emphasise}
	\item[entourer]{circle/surround}
	\item[surligner]{highlight}
	\item[données personnelles]{personal data}
	\item[inscrivez-vous]{register}
	\item[profitez]{enjoy}
	\item[nombreux]{many}
	\item[avantages]{benefits}
	\item[partout]{everywhere}
	\item[naissance]{birth}
	\item[aerobase]{@}
\end{description}
\subsection{Les mois de L'anneé}
\begin{itemize}
	\item Month of the year
	\item Janvier/Février/Mars/Avril
	\item Mai/Juin/Juillet/Août/Septembre
	\item Octobre/Novembre/Décembre
\end{itemize}
\section{04.1 Vocab}
\begin{description}
	\item[aller/je/tu/(il/on/ça)/vous/nous/ils]{go/vais/vas/va/allons/allez/vont}
\end{description}
\begin{description}
	\item[il y a]{There is/there are}
	\item[il n’y a pas de/d’xx]
	\item[aller]{go}
	\item[je vais/tu vas/(il/elle/on/ca) va]
	\item[nous allons/vous allez/(ils/elles) vont]
	\item[un mouchoir]{cloth}
	\item[un mouchoir en papier]{paper}
	\item[une gomme]{eraser}
	\item[un stylo]{pen}
	\item[un ordinateur]{computer}
	\item[un probléme]{problem}
	\item[un crayon]{pencil}
	\item[un parapluie]{umbrella}
	\item[un téléphone]{phone}
	\item[un bureau]{desk}
	\item[un tableau]{painting}
	\item[un ballon]{ball}
	\item[un portefeuilles]{wallet}
	\item[un appareils]{devices}
	\item[des ciseaux]{scissors}
	\item[des lunettes]{glasses}
	\item[des livres]{some books}
	\item[une bouteille]{bottle}
	\item[une chaise]{chair}
	\item[une montre]{watch}
	\item[une règle]{ruler}
	\item[une clé]{key}
	\item[une souris]{mouse}
	\item[une tasse]{cup}
	\item[une glace]{ice}
	\item[trousse]{pencil case}
	\item[ne xx plus de]{no longer/not anymore}
	\item[toujours]{still/everyday}
	\item[maintenant]{now}
	\item[travaille]{work}
	\item[fume]{smoke}
	\item[c’est/ce sont]{it is/they are}
	\item[des]{some}
	\item[doudou]{soft toy}
	\item[d’après une enquête]{according to survey}
	\item[passons]{spend}
	\item[vie]{life}
	\item[cherche]{finding}
	\item[pardons]{lost}
\end{description}
\section{04.2}
\subsubsection{Activities}
\begin{description}
	\item[le footing]{jogging}
	\item[la natation]{swimming}
	\item[la football]{soccer}
	\item[une/l’ escrime]{fencing}
	\item[un box]{boxing}
	\item[le chant]{singing}
	\item[appendre]{learn}
	\item[escalade]{rock climbing}
\end{description}
\subsubsection{Likes}
\begin{description}
	\item[Qu’est-ce que tu preferer]{what do you like(verb)}
	\item[Quelles activités tu aimes]
	\item[aimer/préférer/adorer]{like}
	\item[détester/n’aime pas]{dislike}
\end{description}
\subsubsection{Tb}
\begin{description}
	\item[le sondage]{survey}
	\item[la peinture]{painting}
	\item[la lecture]{to read}
	\item[les bandes dessinées (bayday)]{comic}
	\item[l’opera]{the opera}
	\item[l’elison]{linking vowels tog}
	\item[devient]{becomes}
	\item[devant]{in front}
	\item[mais]{but}
	\item[dors]{sleep}
	\item[à la folie]{to madness}
	\item[faire]{to do}
	\begin{itemize}
		\item je fais/tu fais/(elle/il) fait
		\item vous faites/nous faisons/elles font
	\end{itemize}
	\item[fais xx <activity>]{do <activity>}
	\begin{itemize}
		\item de la(m)/du(f)/de l’/des(plural)
	\end{itemize}
	\item[ne fais pas xx <activity>]{not do..}
	\begin{itemize}
		\item de(all)/d’
	\end{itemize}	
	\item[souvent]{often}
\end{description}
\section{05.1 Time}
\begin{description}
	\item[Quelle heure est-il]{What time is it}
	\item[il est <hr> heures <min>]{its xx time}
	\item[heure]{hour}
	\item[minute]{minutes}
	\item[seconde]{second (segond)}
	\item[5:15]{cinq heures et quart}
	\item[5:30]{“ et demie (ed-mi)}
	\item[5:40]{six heures moins vingt}
	\item[5:50]{“ moins dix}
	\item[5:45]{“ moins le quart}
	\begin{itemize}
		\item Only used from 1-11
		\item Midi or minuit
	\end{itemize}
	\item[tôt]{early}
	\item[tard]{late}
	\item[aujourd’hui]{today}
	\item[demain]{tomorrow}
	\item[hier]{yesterday}
\end{description}
\subsubsection{TB}
\begin{description}
	\item[Une affiche]{poster}
	\item[une poster]{poster}
	\item[autre]{other}
	\item[vouloir]{want}
	\begin{itemize}
		\item je veux/tu veux/(elle/il/on) veut
		\item nous voulons/vous voulez/elles veulent
	\end{itemize}
	\item[aller]{go}
	\begin{itemize}
		\item je vais/tu vas/(il/elle/on) va
		\item vous allez/nous allons/ils vont
	\end{itemize}
	\item[à la bibliothèque(f)]{to xx}
	\item[au (à le) cinéma(m)]
	\item[aux (à les) toilettes(plu)]
	\item[le restaurant]
	\item[le cinéma]
	\item[le théâtre]
	\item[nouveau]{again}
\end{description}
\section{05.2 Vocab}
\begin{description}
	\item[une équerre]{setsquare}
	\item[un marqueur]{marker}
	\item[le clips]
	\item[repérez]{spot}
	\item[mon(m/vowels)/ma(f)]{my}
	\item[ami(m)/amie(f)]{friend}
	\item[un souhait]{wish}
	\item[une voiture]{car}
	\item[perdre]{lose}
	\item[et]{and(never link)}
	\item[main]{hand}
	\item[malade]{sick}
	\item[jardin]{garden}
	\item[quitte]{leaving}
	\item[marionette]{puppet}
\end{description}
\section{06.1 Wishing something to someone}
\begin{description}
	\item[souhaiter]{wishing}
	\item[quelque]{something}
	\item[bonne chance]{good luck}
	\item[bonne soirée]{have a good evening}
	\item[bonne route]{good trip}
	\item[bonne année]{happy new year}
	\item[bon rétablissement]{good recovery}
	\item[bon séjour]{good stay}
	\item[bon courage]{atb}
	\item[entendez]{hear}
	\item[des gens]{people}
	\item[félicitation]{congratulations}
	\item[devoir]{homework}
	\item[sortir]{go out}
	\item[cours]{class}
	\item[dormir]{sleep}
	\item[demander]{ask}
	\item[une équipe]{team}
	\item[festives]{party}
\end{description}
\section{07. On va oû cet ´ét´é}
\begin{description}
	\item[la saisons(f)]{seasons (are masc)}
	\item[le printemps]{spring}
	\item[l’´ét´é]{summer}
	\item[l’automne]{autumn, silent m}
	\item[l’hiver]{winter}
	\item[le soleil]{sun}
	\item[la fleur]{flower}
	\item[il fait beau]{beautiful xx}
	\item[fait \<temperature\>]{temp}
	\item[la neige]{snow ..nayjey..}
	\item[la pluie]{rain}
	\item[le vent]{windy}
	\begin{itemize}
		\item il impersonnel = it/there
	\end{itemize}
	\item[(le) orage]{thunderstorm}
	\item[(le) ´éclairs]{lightning}
	\item[les nuages]{cloudy}
\end{description}
\section{Oral feedback}
\begin{itemize}
	\item e at the end of words are silent
	\item rmb to pronounce some last words
	\begin{itemize}
		\item promena\textbf{d}e
	\end{itemize}
	\item linking vowels together
\end{itemize}
\section{07.2}
\begin{description}
	\item[qu’est-ce que]{what do ..}
	\item[est-ce que]{do you..}
	\begin{itemize}
		\item must be answerd by oui or non
	\end{itemize}
\end{description}
\section{08.2 Lieu (Places)}
\subsection{Continents}
\begin{description}
	\item[




\end{multicols*}
\end{document}

